% !TeX root = main.tex

%%%%%%%%%%%%%%
%% PACKAGES %%
%%%%%%%%%%%%%%


    % DOCUMENT CUSTOMIZATION5

    \usepackage{setspace}
    \usepackage{bm}                     % bm        - Superior bold fonts in math mode.
    \usepackage{enumitem}               % enumitem  - Customization of enumerate and itemize
                                        % environments.
   % \usepackage{hyperref}               % hyperref
%    \usepackage{tocloft}                % tocloft - allows creation of new tables, e.g. list of definitions, list of theorems
                                        % etc. 
                                        % Usage example:
                                        %       \newcommand{\listanswername}{List of Answers}
                                        %       \newlistof[chapter]{answer}{ans}{\listanswername}
                                        %       \newcommand{\answer}[1]{%
                                        %       \refstepcounter{answer}
                                        %       \par\noindent\textbf{Answer \theanswer. #1}
                                        %       \addcontentsline{ans}{answer}{\protect\numberline{\theanswer}#1}\par}

    % SYMBOLS, ENVIRONMENTS, ETC

    \usepackage[colorinlistoftodos,backgroundcolor=green,linecolor=green!50]{todonotes} % Add ``disable in options to disable''
                      


    \usepackage{mathtools}              % mathtools - Exactly what it sounds like, e.g. \xrightarrow.
    \usepackage{subfig}

    
% SYMBOLS, ENVIRONMENTS, ETC.
    \usepackage{graphicx}
    \usepackage{ifthen, xkeyval, xcolor, calc}
    \usepackage{amsmath,amsthm,amssymb} % ams***    - Tons of useful symbols and commands I take for granted.  
    \usepackage{MnSymbol}
    \usepackage{tikz-cd}
    \usepackage{tikz}
        \usetikzlibrary{knots,positioning,decorations.pathreplacing,decorations.pathmorphing,matrix}

% FONTS
    \usepackage{dsfont}                 % dsfont    - I only use this for identity map. Like mathbb with numbers.                                          
    \usepackage{thmtools}
    \usepackage{thm-restate}
    \usepackage%[backend=biber]
        {biblatex}
        %\addbibresource{library.bib}
        \bibliography{library}
    \usepackage{nicematrix}
        
    

%%%%%%%%%%%%%% 
%% COMMANDS %%
%%%%%%%%%%%%%%      
        %\newcommand{\dfn}[1]{\emph{#1}}

        \newcommand{\N}{\mathbb{N}}                 % \N        - Natural Numbers in math mode
        \newcommand{\Z}{\mathbb{Z}}                 % \Z        - The integers in math mode
        \newcommand{\R}{\mathbb{R}}                 % \R        - The Real numbers in math mode
        \newcommand{\Q}{\mathbb{Q}}                 % \Q        - The rational numbers in math mode
        \newcommand{\C}{\mathbb{C}}                 % \C        - The complex numbers in math mode
        
        \newcommand{\dx}{\;dx}
        \newcommand{\dy}{\;dy}
        \newcommand{\dz}{\;dz}

        \DeclareMathOperator{\sr}{sr}
        
        %\newcommand{\a}{\alpha}
        %\newcommand{\b}{\beta}
        
        \newcommand{\gen}[1]{%                      % \gen{}    - Use for generating sets, gives 
                                                    % <stuff>
            \left\langle #1 \right\rangle} 
            \newcommand{\set}[1]{\left\{ #1 \right\}}
            \newcommand{\rk}{\mathrm{rank}~}
        
        
    % MAPS
        \newcommand{\I}{\mathds{1}}                 % \I        - Produces a symbol for the identity map. 
        
    % BRAIDS AND LINKS
            
        %\newcommand{\lk}{\mathrm{lk}}               % \lk       - linking number
        \newcommand{\hc}{\mathcal{H}}               % \hc       - Mathcal H
        \newcommand{\cc}{\mathcal{C}}               % \cc       - Configuration spaces 
        \newcommand{\ccc}{\widetilde{\cc}}           % \ccc      -covering space of \cc
        \newcommand{\fc}{\mathcal{F}}
        \newcommand{\s}{\sigma_}
        \newcommand{\si}{\sigma^{-1}_}
        \newcommand{\g}{\gamma}
        \newcommand{\bi}{\bm{1}_}
    

        \newcommand{\bur}[1][\beta]{\mathrm{bur}_{#1}}
        \newcommand{\bir}[1][\beta]{h_{#1}(x)}
    % MAPPING CLASS GROUPS 
    
        \newcommand{\m}{\mathrm{Mod}}               % \m        - Mod
        \newcommand{\pa}{pseudo-Anosov }
        \newcommand{\dil}[1][\beta]{\lambda_{#1}}
        \newcommand{\mc}{\mathcal{M}}
        
    % Unique to this paper
        \newcommand{\bb}{Band-Boyland~}
        \newcommand{\wh}{\widehat}
        \newcommand{\lk}{\mathrm{Lk}}
        \newcommand{\bmm}{\bm{M}}
        \newcommand{\gm}{G_}
        \newcommand{\mcg}{\mathrm{MCG}}


        \newcommand{\wg}{W(G,g)}

%%%%%%%%%%%%%%%%%%%%%%%%%%%%%
%% CUSTOMIZED ENVIRONMENTS %%
%%%%%%%%%%%%%%%%%%%%%%%%%%%%%
    
    % THEOREM, PROPOSITION, LEMMA, AND COROLLARY
    \theoremstyle{plain}
        \newtheorem{thm}{Theorem}[section]
        \newtheorem{prop}[thm]{Proposition}
        \newtheorem{lem}[thm]{Lemma}
        \newtheorem{cor}[thm]{Corollary}
        \newtheorem{conj}[thm]{Conjecture}

    % DEFINITIONS
        \theoremstyle{definition}
        \newtheorem{dfn}[thm]{{Definition}}
    
    % EXAMPLES
        \theoremstyle{definition}
        \newtheorem{example}[thm]{Example}
        \newtheorem{rmk}{Remark}
          

%%%%%%%%%%%%%%%%%%%%%%%%%%%%%
%% GLOBAL DOCUMENT OPTIONS %%
%%%%%%%%%%%%%%%%%%%%%%%%%%%%%

    % SPACING

        % \onehalfspacing                   % Uncomment to set 1.5 spacing
        % \doublespacing                    % Uncomment to set double spacing
        % \setlength\parindent{0pt}
        % \parskip = 1em                    % Uncomment to set the space between paragraphs.

        %\setlist[enumerate,1]{label=(\alph*)}
        
    % CHAPTER, SECTION, SUBSECTION HEADINGS AND TABLE OF CONTENTS
        %\setcounter{secnumdepth}{1}        % only chapter and sections will be numbered
        %\setcounter{tocdepth}{2}           % entries down to \subsubsections in the TOC
        
    % AUTOREF OPTIONS
        
        %\def\sectionautorefname{section}
        %\def\subsectionautorefname{section}
        %\def\lemautorefname{lemma}
        %\def\thmautorefname{theorem}
        %\def\rmkautorefname{remark}
        %\def\propautorefname{proposition}
        %\def\corautorefname{corollary}

    % Cleveref options
    \usepackage{cleveref}
        \crefname{cor}{corollary}{corrolaries}
        \crefname{thm}{theorem}{theorems}
        \crefname{section}{section}{sections}
        \crefname{subsection}{section}{subsections}
        \crefname{prop}{proposition}{propositions}
        \crefname{lem}{lemma}{lemmas}
        \crefname{figure}{figure}{figures}
        \crefname{dfn}{definition}{definitions}
        \crefname{conj}{conjecture}{conjectures}
        \crefname{rmk}{remark}{remarks}
        \crefname{example}{example}{examples}

            